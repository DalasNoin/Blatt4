\documentclass[11pt]{scrartcl}
\usepackage{graphicx}
\usepackage{graphics}
\usepackage[utf8]{inputenc}
\usepackage[T1]{fontenc}
\usepackage[ngerman]{babel}
\usepackage{svg}
\usepackage{amsmath}
\usepackage{import}

\title{Lösung Blatt 4}
\author{Gruppe Leibniz}

\begin{document}
\maketitle

\section{Aufgabe 1: Medium-Access-Control}
\subsection{Medium-Access-Control (MAC) – Wofür braucht man das?}

Ein Regelwerk, dass festlegt, wann ein Knoten den Funkkanal abhört oder ein Paket verschickt.
Würden die Mitglieder eines Netzwerks beliebig Nachrichten versenden, käme es zu sehr vielen Kollisionen.

\subsection{Warum ist beim Slotted-ALOHA der Datenverlust geringer als beim reinen ALOHA?}

Bei ALOHA können Kollisionen auftreten, wenn sich zwei Nachrichten überschneiden.
Bei Slotted ALOHA kann das nur passieren, wenn zwei Nachrichten genau den gleichen Slot belegen.

\subsection{Beschreiben Sie das Konzept hinter Preamble-Sampling. Welche MAC-Protokolle nutzen dies?}

Ein Netzwerkknoten hört periodisch den Funkkanal ab, ist der Kanal leer, schläft der Knoten für einen bestimmten Zeitraum.
Will ein Knoten etwas versenden, schickt er zuerst eine Lange Preamble, länger als die Schlafdauer.
Dadurch verbrauchen die Empfängerknoten weniger Strom, der Stromverbrauch verschiebt sich zum Sender.
Preamble Sampling wird von X-MAC verwendet.

\subsection{Nennen Sie mindestens drei Unterschiede der Protokolle XMAC und BOXMAC!}



\subsection{Beschreiben Sie das Low-Energy-Adaptive-Clustering-Hierarchy-Verfahren (LEACH) in 3–4 Sätzen!}



\subsection{Beschreiben Sie das Traffic-Adaptive-Medium-Access-Protokoll (TRAMA) in 3–4 Sätzen!}

Um zu entscheiden, ob ein Knoten einen Timeslot verwenden soll, wird global eine Hashfunktion festgelegt.
Die Hashfunktion bildet Zeit t und Node-Identifier x auf eine Zahl ab, die die Priorität darstellt.
Will ein Knoten N eine Nachrichten senden, bestimmt N die eigene Priorität und die aller 2-Hop-Nachbarn, für die nächsten k Zeitschlitze.
N verwendet dann die Zeitschlitze, mit der sie im Vergleich die höchste Priorität hat.

\section{Aufgabe 2: OSI-Modell}

\subsection{Welche Aufgaben haben die sieben Schichten des OSI-Modells jeweils?}

\begin{tabular}{ l l  }
  7 Anwendung & Dateneingabe und -ausgabe \\
  6 Darstellung & Datenkompression, -verschlüsselung, -interpretation \\
  5 Sitzung & Kommunikation eines laufenden Prozesses  \\
  4 Transport & Segmentierung des Datenstroms und die Stauvermittlung  \\
  3 Vermittlung & Schaltung der Verbindungen und die Weitervermittlung von \\ & Datenpaketen  \\
  2 Sicherung & Weitesgehend fehlerfreie Übertragung eines Signals  \\
  1 Bitübertragung & Physikalische übertragung des Signals  \\
\end{tabular}


\subsection{Auf welcher Schicht würden Sie MAC einordnen und warum?}


\end{document}
